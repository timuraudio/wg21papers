\input{wg21common}

\newcommand{\forceindent}{\parindent=1em\indent\parindent=0pt\relax} % For indenting a paragraph containing code that can't be laid out as a {codeblock} because it also contains \emph

\pretolerance=10000 % prevent overfull lines

\begin{document}
\title{Wording for class template argument deduction from inherited constructors}
\author{
  Timur Doumler \small(\href{mailto:papers@timur.audio}{papers@timur.audio})
}
\date{}
\maketitle

\begin{tabular}{ll}
Document \#: & D2582R0 \\
Date: & 2022-04-27\\
Project: & Programming Language C++ \\
Audience: & Core Working Group
\end{tabular}


\begin{abstract}
This paper provides wording for class template argument deduction from inherited constructors \cite{P1021R5}.
\end{abstract}

\section{Proposed wording}

The proposed changes are relative to the C++ working draft \cite{N4910}. 

In [over.match.class.deduct], append to paragraph 1 as follows:

\begin{adjustwidth}{0.5cm}{0.5cm}
\begin{addedblock}
In addition, if \tcode{C} inherits constructors (namespace.udecl) from a base class denoted in the \emph{base-specifier-list} by a \emph{simple-template-id} \tcode{B}, the set contains the functions and function templates formed from an alias template (over.match.alias.deduct) whose template parameters are those of \tcode{C} and whose \emph{simple-template-id} is \tcode{B}.
\end{addedblock}
\end{adjustwidth}

In [over.match.class.deduct], add the following example to the existing block of examples:

\begin{adjustwidth}{0.5cm}{0.5cm}
\begin{addedblock}
\begin{example}
\begin{codeblock}
template <typename T> struct Base {
  Base(T&&);
};

template <typename T> struct Derived : public Base<T> {
  using Base<T>::Base;
}

Derived d(42);  // OK, deduces \tcode{Derived<int>}
\end{codeblock}
\end{example}
\end{addedblock}
\end{adjustwidth}

Insert a bullet after [over.match.best] 1.9 as follows:

\begin{adjustwidth}{0.5cm}{0.5cm}
\begin{addedblock}
\begin{itemize}
\item \tcode{F1} is generated from class template argument deduction (over.match.class.deduct 11.3.1.8) for a class \tcode{D}, \tcode{F2} is generated from inheriting constructors from a base class of \tcode{D}, and for all arguments the corresponding parameters of \tcode{F1} and \tcode{F2} have the same type, or if not that,
\end{itemize}
\end{addedblock}
\end{adjustwidth}


\renewcommand{\bibname}{References}
\bibliographystyle{abstract}
\bibliography{ref}

\end{document}