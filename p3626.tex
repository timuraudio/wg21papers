\input{wg21common}

% Footnotes at bottom of page:
 \usepackage[bottom]{footmisc} 

% Table going across a page: 
 \usepackage{longtable}

 % Start sections at 0
% \setcounter{section}{-1}

% color boxes
\usepackage{tikz,lipsum,lmodern}
\usepackage[most]{tcolorbox}

%%%%%%%%%%%%%%%%%%%%%%%%%%%%%%%%%%%%%%%%%%%%%%%%

%TABLE OF CONTENTS SETTINGS

\usepackage{titlesec}
\usepackage{tocloft}

% Custom ToC layout because the default sucks
\cftsetindents{section}{0in}{0.24in}
\cftsetindents{subsection}{0.24in}{0.34in}
\cftsetindents{subsubsection}{0.58in}{0.44in}

% Needed later to reduce the ToC depth mid document
\newcommand{\changelocaltocdepth}[1]{%
  \addtocontents{toc}{\protect\setcounter{tocdepth}{#1}}%
  \setcounter{tocdepth}{#1}%
}

\setcounter{tocdepth}{3}

%%%%%%%%%%%%%%%%%%%%%%%%%%%%%%%%%%%%%%%%%%%%%%%%

%POLLS

\definecolor{pollFrame}{rgb}{0,.718,0}
\definecolor{pollBG}{rgb}{.5,1,.5}

\newtcolorbox{wgpoll}[1]{colframe=pollFrame,colback=pollBG!20!white,title={#1}}

\newcommand{\wgpollresult}[5]{%

  \vspace{2mm}
  \begin{tabular}{c | c | c | c | c} %
  SF  & F  & N  & A  & SA \\ %
  \hline %
  #1 & #2 & #3 & #4 & #5 \\ %

  \end{tabular}
  \vspace{2mm}  \\ %
}

% Proposal color boxes
\definecolor{proposalBG}{RGB}{147,112,219}

\newcounter{proposalnum}
\newtcbtheorem[use counter=proposalnum]{proposal}
              {Proposal}
              {colback=proposalBG!20!white,colbacktitle={proposalBG!40!white},coltitle={black}}
              {prop}


%%%%%%%%%%%%%%%%%%%%%%%%%%%%%%%%%%%%%%%%%%%%%%%%

\begin{document}
\title{Make predicate exceptions propagate by default}
\author{
Timur Doumler \small(\href{mailto:papers@timur.audio}{papers@timur.audio}) 
}
\date{}
\maketitle

\begin{tabular}{ll}
Document \#: & P3626R0 \\
Date: &2025-02-11 \\
Project: & Programming Language C++ \\
Audience: & EWG
\end{tabular}

\begin{abstract}
One of the concerns over the Contracts for C++ proposal \cite{P2900R13} raised in \cite{P3573R0} is that if a checked contract predicate exits via an exception, that exception is caught and forwarded to the contract-violation handler, which leads to overhead. The suggestion was made that such an exception should instead unconditionally propagate up the stack. This paper provides the necessary wording changes for such a modification.
\end{abstract}

%%%%%%%%%%%%%%%%%%%%%%%%%%%%%%%%%%%%%%%%%%%%%

%\tableofcontents*
%\pagebreak

%\section*{Revision history}

%Revision 0 (2024-04-16)
%\begin{itemize}
%\item Original version
%\end{itemize}

%\pagebreak

%%%%%%%%%%%%%%%%%%%%%%%%%%%%%%%%%%%%%%%%%%%%%

The proposed wording is relative to D2900R14, the draft version that is in CWG and LWG wording review at the time of writing.

Modify [basic.contract.eval] as follows:

\begin{adjustwidth}{0.5cm}{0.5cm}
A \emph{contract violation} occurs when
\begin{itemize}
\item $B$ is \tcode{false},
\item \removed{the evaluation of the predicate exits via an exception, or}
\item the evaluation of the predicate is performed in a context that is manifestly constant-evaluated ([expr.const]) and the predicate is not a core constant expression.
\end{itemize}

\begin{note}
If $B$ is true, no contract violation occurs and control flow continues normally after
the point of evaluation of the contract assertion. \removed{The evaluation of the predicate can fail
to produce a value without causing a contract violation, for example, by calling \tcode{longjmp}
([csetjmp.syn]) or terminating the program.}
\end{note}

\added{If the evaluation of the predicate of a function contract assertion ([dcl.contract.func]) exits via an exception, the behavior is as if the function body exits via that same exception. [ \emph{Note}: A \emph{function-try-block} ([except.pre]) is the function body when present and thus does not have an opportunity to catch the exception. If the function has a non-throwing exception specification, the function \tcode{std::terminate} is invoked ([except.terminate]). --- \emph{end note} ] If the evaluation of the predicate of an \emph{assertion-statement} ([stmt.contract.assert]) exits via an exception, the search for a handler continues from the execution of that statement.}

% TD: editorial note: this entire paragraph is duplicated below for the case where the contract-violation handler exits via an exception, can we deduplicate it somehow?

\begin{note}
The\added{re are other circumstances in which the} evaluation of the predicate can fail
to produce a value without causing a contract violation, for example, by calling \tcode{longjmp}
([csetjmp.syn]) or terminating the program.
\end{note}

\phantom{x}

[...]

\phantom{x}

\removed{ If the contract violation occurred because the evaluation of the predicate exited via an
exception, the contract-violation handler is invoked from within an active implicit handler
for that exception ([except.handle]). If the contract-violation handler returns normally
and the evaluation semantic is observe, that implicit handler is no longer considered
active.}

\removed{[ \emph{Note:} The exception can be inspected or rethrown within the contract-violation handler. --- \emph{end note} ]}

If the contract-violation handler returns normally and the evaluation semantic is enforce,
the program is contract-terminated\removed{; if violation occurred as the result of an uncaught
exception from the evaluation of the predicate, the implicit handler remains active when
contract termination occurs}.

\end{adjustwidth}

Modify [except.terminate] as follows:

\begin{adjustwidth}{0.5cm}{0.5cm}
Some errors in a program cannot be recovered from, such as when an exception is not
handled or a \tcode{std::thread} object is destroyed while its thread function is still executing.
In such cases, the function \tcode{std::terminate} ([exception.terminate]) is invoked. 
\begin{note}
These situations are:
\begin{itemize}
\item ...
\item when \added{evaluating the predicate of a function contract assertion ([dcl.contract.func]) or} a contract-violation handler ([basic.contract.handler]) invoked from evaluating a function contract assertion on a function with a non-throwing exception specification exits via an exception, or
\item ...
\end{itemize}
\end{note}
\end{adjustwidth}

Modify [contracts.syn] as follows:

\begin{adjustwidth}{0.5cm}{0.5cm}
% TD: TODO: If I put the underscore inside \tcode it messes everything up :(
\removed{\tcode{enum class detection}_\tcode{mode : \emph{\tcode{unspecified}} \{}} \\
\removed{\phantom{\tcode{xx}}\tcode{predicate}_\tcode{false = 1,}} \\
\removed{\phantom{\tcode{xx}}\tcode{evaluation}_\tcode{exception = 2}} \\
\removed{\tcode{\};}}

\phantom{x}

[...]

\phantom{x}

\tcode{class contract_violation \{} \\
\phantom{\tcode{xx}} \emph{// no user-accessible constructor} \\
public: \\
\phantom{\tcode{xx}} \tcode{contract_violation(const contract_violation\&) = delete;} \\
\phantom{\tcode{xx}} \tcode{contract_violation\& operator=(const contract_violation\&) = delete;}  \\
\\
\phantom{\tcode{xx}} \emph{/* see below */} \tcode{~contract_violation();} \\
\\
\phantom{\tcode{xx}} \tcode{const char* comment() const noexcept}; \\
\phantom{\tcode{xx}} \tcode{detection_mode detection_mode() const noexcept;} \\
\phantom{\tcode{xx}} \removed{\tcode{exception}_\tcode{ptr evaluation}_\tcode{exception() const noexcept;}} \\
\phantom{\tcode{xx}} \tcode{bool is_terminating() const noexcept;} \\
\phantom{\tcode{xx}} \tcode{assertion_kind kind() const noexcept;} \\
\phantom{\tcode{xx}} \tcode{source_location location() const noexcept;} \\
\phantom{\tcode{xx}} \tcode{evaluation_semantic semantic() const noexcept;} \\
\tcode{\};}
\end{adjustwidth}

Remove [support.contract.enum.detection] entirely.

Modify [support.contract.violation] as follows:

\begin{adjustwidth}{0.5cm}{0.5cm}
\removed{\tcode{contracts::detection}_\tcode{mode detection}_\tcode{mode() const noexcept;}} 

\removed{\emph{Returns:} The enumerator value corresponding to the manner in which the contract violation was identified.}

\phantom{x}

[...]

\phantom{x}

\removed{\tcode{exception}_\tcode{ptr evaluation}_\tcode{exception() const noexcept;}} 

\removed{\emph{Returns:} If the contract violation occurred because the evaluation of the predicate exited via an exception, an \tcode{exception}_\tcode{ptr} object that refers to that exception or a copy of that exception; otherwise, a null \tcode{exception}_\tcode{ptr} object.}
\end{adjustwidth}

%%%%%%%%%%%%%%%%%%%%%%%%%%%%%%%%%%%%%%%%%%%%%

%\section*{Acknowledgements}
%Thanks to Oliver Rosten for his review of the paper.

%%%%%%%%%%%%%%%%%%%%%%%%%%%%%%%%%%%%%%%%%%%%%

% Remove ToC entry for bibliography
\renewcommand{\addcontentsline}[3]{}% Make \addcontentsline a no-op to disable auto ToC entry

%\renewcommand{\bibname}{References}  % custom name for bibliography
\bibliographystyle{abstract}
\bibliography{ref}

%%%%%%%%%%%%%%%%%%%%%%%%%%%%%%%%%%%%%%%%%%%%%


\end{document}
