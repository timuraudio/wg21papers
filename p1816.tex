\input{wg21common}

\newcommand{\forceindent}{\parindent=1em\indent\parindent=0pt\relax} % For indenting a paragraph containing code that can't be laid out as a {codeblock} because it also contains \emph

\pretolerance=10000 % prevent overfull lines

\begin{document}
\title{Wording for class template argument deduction forr aggregates}
\author{
  Timur Doumler \small(\href{mailto:papers@timur.audio}{papers@timur.audio})
}
\date{}
\maketitle

\begin{tabular}{ll}
Document \#: & P1816R0 \\
Date: & 2019-07-17\\
Project: & Programming Language C++ \\
Audience: & Core Working Group
\end{tabular}


\begin{abstract}
This paper provides wording for class template argument deduction for aggregates \cite{P1021R4}.
\end{abstract}

\section{Proposed wording}

The proposed changes are relative to the C++ working draft \cite{N4820}. 

In [over.match.class.deduct], append to paragraph 1 as follows:

\begin{adjustwidth}{0.9cm}{0.9cm}
\begin{itemize}
\item For each \grammarterm{deduction-guide}, a function or function template with the following properties:
\begin{itemize}
\item The template parameters, if any, and function parameters are those of the \grammarterm{deduction-guide}.
\item The return type is the \grammarterm{simple-template-id} of the \grammarterm{deduction-guide}.
\end{itemize}
\end{itemize}

\begin{addedblock}
In addition, if \tcode{C} satisfies the conditions for an aggregate class with the assumption that any dependent base class has no virtual functions and no virtual base classes, and the initializer is a non-empty \grammarterm{braced-init-list} or parenthesized \grammarterm{expression-list}, the set contains an additional function template, called the \grammarterm{aggregate deduction candidate}, defined as follows. Let $x_1, ..., x_n$ be the elements of the \grammarterm{initializer-list} or \grammarterm{designated-initializer-list} of the \grammarterm{braced-init-list}, or of the \grammarterm{expression-list}. For each $x_i$, let $e_i$ be the corresponding element of \tcode{C} or of one of its (possibly recursive) subaggregates that would be initialized by $x_i$ ([dcl.init.aggr]) if brace elision is not considered for any subaggregate that has a dependent type. If there is no such element $e_i$, the program is ill-formed. The aggregate deduction candidate is derived as above from a hypothetical constructor $\tcode{C}(\tcode{T}_1, ..., \tcode{T}_n)$, where $\tcode{T}_i$ is the declared type of the element $e_i$.
\end{addedblock}
\end{adjustwidth}

In [over.match.class.deduct], paragraph 3, add to the example as follows:

\begin{adjustwidth}{0.5cm}{0.5cm}
\begin{codeblock}
B b{(int*)0, (char*)0};      // OK, deduces \tcode{B<char*>}
\end{codeblock}
\begin{addedblock}
\begin{codeblock}
template <typename T>
struct S {
    T x;
    T y; 
};

template <typename T>
struct C {
    S<T> s;
    T t; 
};

template <typename T>
struct D {
    S<int> s;
    T t; 
};

C c1 = {1, 2};         // error: deduction failed 
C c2 = {1, 2, 3};      // error: deduction failed 
C c3 = {{1u, 2u}, 3};  // OK, \tcode{C<int>} deduced

D d1 = {1, 2};         // error: deduction failed
D d2 = {1, 2, 3};      // OK, braces elided, \tcode{D<int>} deduced

template <typename T>
struct I {
    using type = T;
};

template <typename T>
struct E {
    typename I<T>::type i;
    T t; 
};

E e1 = {1, 2};          // OK, \tcode{E<int>} deduced
\end{codeblock}
\end{addedblock}
---  \emph{end example} ]
\end{adjustwidth}


\renewcommand{\bibname}{References}
\bibliographystyle{abstract}
\bibliography{ref}

\end{document}