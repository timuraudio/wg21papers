\input{wg21common}

\newcommand{\forceindent}{\parindent=1em\indent\parindent=0pt\relax} % For indenting a paragraph containing code that can't be laid out as a {codeblock} because it also contains \emph

\pretolerance=10000 % prevent overfull lines

\begin{document}
\title{Wording for class template argument deduction from inherited constructors}
\author{
  Timur Doumler \small(\href{mailto:papers@timur.audio}{papers@timur.audio})
}
\date{}
\maketitle

\begin{tabular}{ll}
Document \#: & D2582R1 \\
Date: & 2022-05-17\\
Project: & Programming Language C++ \\
Audience: & Core Working Group
\end{tabular}


\begin{abstract}
This paper provides wording for class template argument deduction from inherited constructors \cite{P1021R6}.
\end{abstract}

\section{Proposed wording}

The proposed changes are relative to the C++ working draft \cite{N4910}. 

In [over.match.class.deduct], append to paragraph 1 as follows:

\begin{adjustwidth}{0.5cm}{0.5cm}
except that additional parameter packs of the form $\tcode{P}_j \tcode{...}$
are inserted into the parameter list in their original aggregate element position corresponding to each non-trailing aggregate element of type $\tcode{P}_j$
that was skipped because it was a parameter pack, and
the trailing sequence of parameters corresponding
to a trailing aggregate element that is a pack expansion (if any)
is replaced by a single parameter of the form $\tcode{T}_n \tcode{...}$.

\added{In addition, if \tcode{C} is defined and inherits constructors ([namespace.udecl]) from a direct base class denoted in the \emph{base-specifier-list} by a \emph{class-or-decltype} \tcode{B}, the set contains the guides of an alias template \tcode{A} whose template parameter list is that of \tcode{C} and whose \emph{defining-template-id} is \tcode{B} if \tcode{A} is a deducible template ([dcl.type.simple]), as described above, except that the return type is the class template specialization designated by \tcode{C} and template arguments corresponding to the template parameters of \tcode{B}.}
\end{adjustwidth}

In [over.match.class.deduct], add the following example:

\begin{adjustwidth}{0.5cm}{0.5cm}
\begin{addedblock}
\begin{example}
\begin{codeblock}
template <typename T> struct B {
  B(T);
};

template <typename T> struct C : public B<T> {
  using B<T>::B;
};

template <typename T> struct D : public B<T> {};

C c(42);  // OK, deduces \tcode{C<int>}
D d(42);  // Error: deduction failed, no inherited deduction guides

B(int) -> B<char>;
C c2(42); // OK, deduces \tcode{C<char>}

template <typename T> struct E : public B<int> {
  using B<int>::B;
};

E e(42);  // Error: deduction failed, base class is not a deducible template

template <typename T, typename U, typename V> struct F {
  F(T, U, V);
};

template <typename T, typename U> struct G : F<U, T, int> {
  using F<U, T, int>::F;
}

G g(true, 'a', 1); // OK, deduces \tcode{G<char, bool>}

\end{codeblock}
\end{example}
\end{addedblock}
\end{adjustwidth}

In [over.match.best.general], insert as follows:

\begin{adjustwidth}{0cm}{0.5cm}
\begin{itemize}
\item
\tcode{F1} and \tcode{F2} are rewritten candidates, and
\tcode{F2} is a synthesized candidate
with reversed order of parameters
and \tcode{F1} is not
\begin{example}
\begin{codeblock}
struct S {
  friend std::weak_ordering operator<=>(const S&, int);         // \#1
  friend std::weak_ordering operator<=>(int, const S&);         // \#2
};
bool b = 1 < S();                                               // calls \#2
\end{codeblock}
\end{example}
or, if not that,


\item \added{\tcode{F1} and \tcode{F2} are generated from class template argument deduction ([over.match.class.deduct]) for a class \tcode{D}, and \tcode{F2} is generated from inheriting constructors from a base class of \tcode{D} while \tcode{F1} is not, and for each explicit function argument, the corresponding parameters of \tcode{F1} and \tcode{F2} are either both ellipses or have the same type, or, if not that,}

\item
\tcode{F1} is generated from a
\emph{deduction-guide} ([over.match.class.deduct])
and \tcode{F2} is not, or, if not that,
\end{itemize}
\end{adjustwidth}

%%%%%%%%%%%%%%%%%%%%%%%%%%%%%%%%

\section*{Document history}

\begin{itemize}
\item \textbf{R0}, 2022-05-15: Initial version.
\item \textbf{R1}, 2022-05-17: Various wording fixes.
\end{itemize}

%%%%%%%%%%%%%%%%%%%%%%%%%%%%%%%%

\section*{Acknowledgements}

Many thanks to Hubert Tong for his help with fixing the wording.


\renewcommand{\bibname}{References}
\bibliographystyle{abstract}
\bibliography{ref}

\end{document}