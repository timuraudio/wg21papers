\input{wg21common}

% Footnotes at bottom of page:
 \usepackage[bottom]{footmisc} 

% Table going across a page: 
 \usepackage{longtable}

 % Start sections at 0
% \setcounter{section}{-1}

% color boxes
\usepackage{tikz,lipsum,lmodern}
\usepackage[most]{tcolorbox}

%%%%%%%%%%%%%%%%%%%%%%%%%%%%%%%%%%%%%%%%%%%%%%%%

%TABLE OF CONTENTS SETTINGS

\usepackage{titlesec}
\usepackage{tocloft}

% Custom ToC layout because the default sucks
\cftsetindents{section}{0in}{0.24in}
\cftsetindents{subsection}{0.24in}{0.34in}
\cftsetindents{subsubsection}{0.58in}{0.44in}

% Needed later to reduce the ToC depth mid document
\newcommand{\changelocaltocdepth}[1]{%
  \addtocontents{toc}{\protect\setcounter{tocdepth}{#1}}%
  \setcounter{tocdepth}{#1}%
}

\setcounter{tocdepth}{3}

%%%%%%%%%%%%%%%%%%%%%%%%%%%%%%%%%%%%%%%%%%%%%%%%

%POLLS

\definecolor{pollFrame}{rgb}{0,.718,0}
\definecolor{pollBG}{rgb}{.5,1,.5}

\newtcolorbox{wgpoll}[1]{colframe=pollFrame,colback=pollBG!20!white,title={#1}}

\newcommand{\wgpollresult}[5]{%

  \vspace{2mm}
  \begin{tabular}{c | c | c | c | c} %
  SF  & F  & N  & A  & SA \\ %
  \hline %
  #1 & #2 & #3 & #4 & #5 \\ %

  \end{tabular}
  \vspace{2mm}  \\ %
}

%%%%%%%%%%%%%%%%%%%%%%%%%%%%%%%%%%%%%%%%%%%%%%%%

\begin{document}
\title{Contracts for C++: Wroc\l aw technical fixes}
\author{
Timur Doumler \small(\href{mailto:papers@timur.audio}{papers@timur.audio}) \\
Joshua Berne \small(\href{mailto:jberne4@bloomberg.net}{jberne4@bloomberg.net}) \\
Andrzej Krzemie\' nski \small(\href{mailto:akrzemi@gmail.com}{akrzemi@gmail.com}) \\
}
\date{}
\maketitle

\begin{tabular}{ll}
Document \#: & D3520R0 \\
Date: &2024-11-21 \\
Project: & Programming Language C++ \\
Audience: & SG21, EWG
\end{tabular}

\begin{abstract}
During CWG wording review of Contracts \cite{P2900R11} at the November 2024 Wroc\l aw meeting a number of minor issues were brought up. This paper discusses them and proposes resolutions.
\end{abstract}

%%%%%%%%%%%%%%%%%%%%%%%%%%%%%%%%%%%%%%%%%%%%%

%\tableofcontents*
%\pagebreak

%\section*{Revision history}

%Revision 0 (2024-04-16)
%\begin{itemize}
%\item Original version
%\end{itemize}

%\pagebreak

%%%%%%%%%%%%%%%%%%%%%%%%%%%%%%%%%%%%%%%%%%%%%

\section{Specify the mode of termination}

The current wording in \cite{P2900R11} says that if a contract violation occurs while evaluating a contract assertion with the enforce or quick-enforce semantic, ``the program is terminated in an implementation-defined fashion''.

According to CWG, this specification is insufficient to determine which modes of termination are conforming. CWG is asking to enumerate the specific modes of termination that a conforming implementation is allowed to choose between.

The following modes of termination exist in C++ today (ordered by severity):

\begin{enumerate}
\item \tcode{std::exit} --- Normal program termination with full cleanup
\item \tcode{std::quick_exit} --- Normal program termination, but with less cleanup
\item \tcode{std::_Exit} --- Normal program termination with minimal cleanup
\item \tcode{std::terminate} --- Abnormal program termination, the C++ way
\item \tcode{std::abort} --- Abnormal program termination, the C way
\item \tcode{__builtin_trap}, \tcode{__builtin_verbose_trap}, \tcode{ __fastfail}, \tcode{__debugbreak}, etc. --- Implementation-defined abnormal termination modes with no cleanup
\end{enumerate}

These modes of termination perform the following actions (a question mark denotes that it is implementation-defined whether the action happens):

%%%%%%%%%%%%%%%%%%%%%%%%%%%%%%%%%%%%%%%%%%%%%
\pagebreak
%%%%%%%%%%%%%%%%%%%%%%%%%%%%%%%%%%%%%%%%%%%%%

%%%%%%%%%%%%%%%%%%%%%%%%%%%%%%%%%%%%%%%%%%%%%
\newcommand{\yes}{\includegraphics[width=4mm]{images/yes.png}}
\newcommand{\no}{\includegraphics[width=4mm]{images/no.png}}
\newcommand{\maybe}{\includegraphics[width=4mm]{images/maybe.png}}
%\vspace{4mm}
\begin{table}[h!]
\begin{tabular}{|p{8cm}|p{0.9cm}|p{0.9cm}|p{0.9cm}|p{0.9cm}|p{0.9cm}|p{0.9cm}|}
\hline 
& 1 & 2 & 3 & 4 & 5 & 6 \\
\hline
Can be used for normal termination that returns \tcode{0} to the host environment & \yes & \yes  & \yes  & \no & \no & \no\\ \hline
Calls destructors of static and thread local objects & \yes & \no  & \no  & \no & \no & \no\\ \hline
Calls callback functions that can be defined by the user & \yes & \yes  & \no  & \yes & \yes & \no\\ \hline
Flushes and closes streams, removes temporary files & \yes & \maybe  & \maybe  & \maybe & \maybe & \no\\ \hline
\end{tabular}
\end{table}
%%%%%%%%%%%%%%%%%%%%%%%%%%%%%%%%%%%%%%%


The callback functions called vary by each termination mode. \tcode{std::exit} calls functions registered with \tcode{std::atexit}. \tcode{std::quick_exit} calls functions registered with \tcode{std::at_quick_exit}. \tcode{std::terminate} calls the currently installed \tcode{std::termination_handler}; the default handler calls \tcode{std::abort}. \tcode{std::abort} raises a \tcode{SIGABRT} signal which may be caught by an appropriate signal handler.

\tcode{std::exit} is called when \tcode{main} returns. \tcode{std::_Exit} is called by \tcode{std::exit} and \tcode{std::quick_exit} after they have performed their respective cleanup actions. \tcode{std::terminate} is usually called when an unrecoverable error occurs during exception handling, but has a few other notable use cases, e.g., destroying a joinable \tcode{std::thread}. \tcode{std::abort} is called by a failing \tcode{cassert}. All termination modes can be directly invoked by the user. All termination modes except 6 require calling into a C or C++ standard library function.

In all cases, termination after a contract violation results from a \emph{defect} in the program; therefore, only the abnormal modes of termination seem appropriate. The following modes of termination (which are all abnormal modes) are either already used by existing implementations of \cite{P2900R11} or were requested as options by compiler vendors:

\begin{itemize}
\item \tcode{std::terminate} --- used by the Clang implementation for quick-enforce when predicate evaluation exits via an exception;
\item \tcode{std::abort} --- to make the default behaviour of a contract violation identical to that of a failed \tcode{cassert};
\item as-if by \tcode{std::abort} --- requested by a compiler vendor in order to be able to raise a \tcode{SIGABRT} signal without having to call a standard library function ;
\item \tcode{__builtin_trap} and \tcode{__builtin_debug_trap} --- used by the Clang implementation for quick-enforce when the predicate evaluates to \tcode{false} to minimize code size and the amount of instructions executed after a contract violation.
\end{itemize}

Therefore, we propose to allow exactly these options. Proposed wording:

\begin{adjustwidth}{0.5cm}{0.5cm}
\begin{addedblock}
When the program is \emph{contract-terminated}, depending on context:
\begin{itemize}
\item \tcode{std::terminate} is called,
\item \tcode{std::abort} is called,
\item execution is terminated as-if by performing the operations of \tcode{std::abort},
\item execution is terminated.
\end{itemize}
\end{addedblock}

If a contract violation occurs in a context that is not manifestly constant-evaluated and the evaluation semantic is quick-enforce, the program is \removed{terminated in an implementation-defined fashion}\added{contract-terminated}.

If a contract violation occurs in a context that is not manifestly constant-evaluated and the evaluation semantic is enforce, [...] If the contract-violation handler returns normally and the evaluation semantic is enforce, the program is \removed{terminated in an implementation-defined fashion}\added{contract-terminated}.
\end{adjustwidth}

\section{Disallow lambdas inside contract redeclarations}

sdfgd

\section{Unify rules for parameters of dependent type}

sdf
%%%%%%%%%%%%%%%%%%%%%%%%%%%%%%%%%%%%%%%%%%%%%

%\section*{Acknowledgements}
%Thanks to Oliver Rosten for his review of the paper.

%%%%%%%%%%%%%%%%%%%%%%%%%%%%%%%%%%%%%%%%%%%%%

% Remove ToC entry for bibliography
\renewcommand{\addcontentsline}[3]{}% Make \addcontentsline a no-op to disable auto ToC entry

%\renewcommand{\bibname}{References}  % custom name for bibliography
\bibliographystyle{abstract}
\bibliography{ref}

%%%%%%%%%%%%%%%%%%%%%%%%%%%%%%%%%%%%%%%%%%%%%


\end{document}
