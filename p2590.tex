\input{wg21common}

\newcommand{\forceindent}{\parindent=1em\indent\parindent=0pt\relax} % For indenting a paragraph containing code that can't be laid out as a {codeblock} because it also contains \emph

\begin{document}
\title{Direct object creation with \texttt{std::start_lifetime_as}}
\author{
  Timur Doumler \small(\href{mailto:papers@timur.audio}{papers@timur.audio}) \\
  Richard Smith \small(\href{mailto:richardsmith@google.com}{richardsmith@google.com})
}
\date{}
\maketitle

\begin{tabular}{ll}
Document \#: & P2590R0 \\
Date: & 2022-05-15\\
Project: & Programming Language C++ \\
Audience: & Library Working Group, Core Working Group
\end{tabular}


\begin{abstract}
This paper proposes a new standard library facility \tcode{std::start_lifetime_as}. For objects of sufficiently trivial types, this facility can be used to directly create objects and start their lifetime on-demand to give programs defined behaviour. This proposal completes the functionality originally proposed in \cite{P0593R6} by providing the standard library portion of that paper (only the core language portion of that paper made it into C++20).
\end{abstract}

\vspace{5mm}

\section{Motivation}
\label{sec:motivation}

TODO

%%%%%%%%%%%%%%%%%%%%%%%%%%%%%%%%

\section{History}
\label{sec:history}

TODO

%%%%%%%%%%%%%%%%%%%%%%%%%%%%%%%%

\section{Proposed wording}
\label{sec:wording}

TODO

%%%%%%%%%%%%%%%%%%%%%%%%%%%%%%%%

%\section*{Document history}

%\begin{itemize}
%\item \textbf{R0}, 2022-05-15: Initial version.
%\end{itemize}

%%%%%%%%%%%%%%%%%%%%%%%%%%%%%%%%

%\section*{Acknowledgements}

%Nothing yet

%%%%%%%%%%%%%%%%%%%%%%%%%%%%%%%%

\renewcommand{\bibname}{References}
\bibliographystyle{abstract}
\bibliography{ref}

\end{document}